\documentclass[]{article}
\usepackage{lmodern}
\usepackage{amssymb,amsmath}
\usepackage{ifxetex,ifluatex}
\usepackage{fixltx2e} % provides \textsubscript
\ifnum 0\ifxetex 1\fi\ifluatex 1\fi=0 % if pdftex
  \usepackage[T1]{fontenc}
  \usepackage[utf8]{inputenc}
\else % if luatex or xelatex
  \ifxetex
    \usepackage{mathspec}
  \else
    \usepackage{fontspec}
  \fi
  \defaultfontfeatures{Ligatures=TeX,Scale=MatchLowercase}
\fi
% use upquote if available, for straight quotes in verbatim environments
\IfFileExists{upquote.sty}{\usepackage{upquote}}{}
% use microtype if available
\IfFileExists{microtype.sty}{%
\usepackage{microtype}
\UseMicrotypeSet[protrusion]{basicmath} % disable protrusion for tt fonts
}{}
\usepackage[margin=1in]{geometry}
\usepackage{hyperref}
\hypersetup{unicode=true,
            pdftitle={Technical Report},
            pdfauthor={Jordan Majoros, Alli McCarty, Joe Spatz},
            pdfborder={0 0 0},
            breaklinks=true}
\urlstyle{same}  % don't use monospace font for urls
\usepackage{graphicx}
% grffile has become a legacy package: https://ctan.org/pkg/grffile
\IfFileExists{grffile.sty}{%
\usepackage{grffile}
}{}
\makeatletter
\def\maxwidth{\ifdim\Gin@nat@width>\linewidth\linewidth\else\Gin@nat@width\fi}
\def\maxheight{\ifdim\Gin@nat@height>\textheight\textheight\else\Gin@nat@height\fi}
\makeatother
% Scale images if necessary, so that they will not overflow the page
% margins by default, and it is still possible to overwrite the defaults
% using explicit options in \includegraphics[width, height, ...]{}
\setkeys{Gin}{width=\maxwidth,height=\maxheight,keepaspectratio}
\IfFileExists{parskip.sty}{%
\usepackage{parskip}
}{% else
\setlength{\parindent}{0pt}
\setlength{\parskip}{6pt plus 2pt minus 1pt}
}
\setlength{\emergencystretch}{3em}  % prevent overfull lines
\providecommand{\tightlist}{%
  \setlength{\itemsep}{0pt}\setlength{\parskip}{0pt}}
\setcounter{secnumdepth}{0}
% Redefines (sub)paragraphs to behave more like sections
\ifx\paragraph\undefined\else
\let\oldparagraph\paragraph
\renewcommand{\paragraph}[1]{\oldparagraph{#1}\mbox{}}
\fi
\ifx\subparagraph\undefined\else
\let\oldsubparagraph\subparagraph
\renewcommand{\subparagraph}[1]{\oldsubparagraph{#1}\mbox{}}
\fi

%%% Use protect on footnotes to avoid problems with footnotes in titles
\let\rmarkdownfootnote\footnote%
\def\footnote{\protect\rmarkdownfootnote}

%%% Change title format to be more compact
\usepackage{titling}

% Create subtitle command for use in maketitle
\providecommand{\subtitle}[1]{
  \posttitle{
    \begin{center}\large#1\end{center}
    }
}

\setlength{\droptitle}{-2em}

  \title{Technical Report}
    \pretitle{\vspace{\droptitle}\centering\huge}
  \posttitle{\par}
    \author{Jordan Majoros, Alli McCarty, Joe Spatz}
    \preauthor{\centering\large\emph}
  \postauthor{\par}
      \predate{\centering\large\emph}
  \postdate{\par}
    \date{2/15/2021}

\usepackage{booktabs}
\usepackage{longtable}
\usepackage{array}
\usepackage{multirow}
\usepackage{wrapfig}
\usepackage{float}
\usepackage{colortbl}
\usepackage{pdflscape}
\usepackage{tabu}
\usepackage{threeparttable}
\usepackage{threeparttablex}
\usepackage[normalem]{ulem}
\usepackage{makecell}

\begin{document}
\maketitle

\hypertarget{abstract}{%
\subsection{Abstract:}\label{abstract}}

The purpose of this analysis is to use modern statistical applications
to determine what variables, with a particular emphasis on minutes of
metabolic exercise and time of work shift, influence weight gain in
Americans. The data comes from a self-report questionnaire administered
by a call center over an eight month period. The data was processed and
exploratory data analysis was performed to gain insight into basic
trends and collinearity between variables. Several transformations were
performed on variables with skewed distributions, and the most important
variables were selected via nonparametric analysis to be included in our
models. Two models were constructed; a logistic regression model to
obtain binary predictions on weight gain, and a generalized additive
model using various predictors to estimate weight gain in pounds. The
model fits were cross validated via **\emph{WHAT??} and residual
analysis was performed.

\hypertarget{introduction}{%
\subsection{Introduction:}\label{introduction}}

Diseases such as obesity, diabetes, coronary heart disease, and
musculoskeletal disorders have rapidly increased in the United States in
the last decade, which can be partially attributed to the change in
office culture to be heavily computer-based and therefore more
sedentary. Increased prevalence of such diseases are expensive to
organizations and influence how productive and efficient workers can be.
Lifestyle habits of employees affect rates of obesity and
obesity-related diseases in the United States. The purpose of our
analysis was to answer what is affecting the weight change of employees
in a call center. Descriptive statistics and graphs were used to
visualize relationships between different categorical and continuous
variables. We confirmed normality for these variables and used log plus
one transformation for right skewed distributions. Work shift and total
metabolic minutes were two variables at the center of our analysis. We
showed their effect on weight gain through exploratory data analysis.
Two models have been built to show the statistical significance of
certain predictors. The first model used logistic regression with weight
gain (yes/no) being the categorical response variable. Our second model
was a multivariate linear model that focuses on the significance of
shift start time and total metabolic minutes. The following report
records our central ideas and provides evidence for the procedures
taken.

\hypertarget{data}{%
\subsection{Data:}\label{data}}

Three additional variables were calculated using the given data and
added to the dataset to give more depth to BMI. The participants of the
call center study provided their height, number of pounds gained during
the eight month period, and body weight at the end as well. With those
values, the ending BMI was calculated. To make the ending BMI more
valuable we used the provided data to calculate the BMI for each
participant before starting the 8 month study. First, the beginning body
weight was found using the pounds gained and the weight of the
participant at the end. Then the starting BMI was calculated using the
formula below with weight measured in pounds and the height the
participant presented in inches.

\begin{verbatim}
            BMI = 703 * weight (lbs) / [height (inches)]^2
            
\end{verbatim}

Change in BMI was then calculated subtracting the starting BMI from the
ending BMI. Many participants did not gain weight so their change in BMI
was zero. Percentage BMI change over the 8 month period could have also
been calculated by dividing the BMI difference and beginning BMI but
that was not included as an additional variable. Analyzing the change in
BMI can have advantages over strictly pounds gain if a participant
increased their height or started at a below average weight given they
are taller than average.

\hypertarget{exploratory-data-analysis}{%
\subsection{Exploratory Data
Analysis:}\label{exploratory-data-analysis}}

In order to test the assumption of normalcy for the variables, we made
QQPlots for all of the continuous numeric variables (Appendix 1). Based
on the plots, we think that age, height, body weight, and BMI are
approximately normal, at least normal enough that they do not merit
transformations.

Pounds gained, VigTime, ModTime, and WalkTime, and TotalMetMinutes do
not follow a normal distribution and have a strong right skew. We will
take a closer look at these variables and try a log transformation if
necessary.

\includegraphics{test_latex_files/figure-latex/unnamed-chunk-1-1.pdf}

\includegraphics{test_latex_files/figure-latex/unnamed-chunk-2-1.pdf}

\hypertarget{visuals-for-categorical-variables}{%
\subsubsection{Visuals for Categorical
Variables}\label{visuals-for-categorical-variables}}

Our first graph shows the distribution of total metabolic minutes by
increasing age for both genders of employees in the call center study.
Female employees see most of their peak total metabolic minutes from the
age group 20 to 30. After around age 32 it begins to average out and see
a smaller range of values. From age 45 onward the total metabolic
minutes stays under 2400 minutes. The male employees have their total
metabolic minutes peak just under 7500 minutes. The rest of the male
call center employees had total metabolic minutes under about 3500
minutes. Male and female employees do not show similar total metabolic
minutes distributions for ages 20 to 60.

\includegraphics{test_latex_files/figure-latex/unnamed-chunk-3-1.pdf}

The change in BMI over the 8 month study period for the different shift
start times is displayed in a boxplot with male and female separated. We
see smaller spread of BMI change values for work shifts starting between
11am and 2pm with only one female observation at the 2pm shift being a
large outlier. The 8am start time shift is the only time that has a more
spread distribution of BMI change for male over female employees. There
are less male employees so the mean BMI change for each shift can be
pulled higher more easily by the outlier observations. Overall we see
larger BMI changes from the morning shifts of both genders.

\includegraphics{test_latex_files/figure-latex/unnamed-chunk-4-1.pdf}
The next four visuals show average exercise time in minutes by gender
for different intensity exercises. The three intensity measures are
vigorous, moderate, and walking. Vigorous being the high-intensity
exercise. The fourth barplot uses the three intensities to calculate
total metabolic minutes. Each graph examines the average times for each
starting shift time.

\includegraphics{test_latex_files/figure-latex/unnamed-chunk-5-1.pdf}

Now, let's test for collinearity between variables:

\includegraphics{test_latex_files/figure-latex/unnamed-chunk-6-1.pdf}

The first obvious thing to note is that BMI and body weight are
positively correlated. This should be expected, because we caluclated
BMI via a linear combination including body weight.

\hypertarget{model-process}{%
\subsection{Model Process:}\label{model-process}}

Commentary on lack of data for lbs gained and motivation for building
two separate models:

There are a lot of N/A values for lbs gained because we will not have
data for a specific observation unless they gained weight. If a person
lost weight in the study, this would be populated as ``N/A'' for lbs
gained (col 6). For the purposes of the second model, we may choose to
select the subset of the population who gained weight (have data for
this variable)/or did not gain weight (do NOT have data for this
variable) and create a model specific to that subset.

For Model 1, we will just use the binary weight gain variable to create
a logistic model. For Model 2, we will subset the sample to those who
gained weight, and build multiple models to determine which factors
influenced this weight gain.

Data Cleaning specific to our models:

We created a logistic regression model for our binary dependent variable
`weightgain', yes or no. The model with the best fit contained three
predictor variables: Total\_Met\_Min, gender, and Age. The response will
be close to 1 when the person has gained weight and 0 when they have not
gained weight over the 8 month period.

glm(weightgain \textasciitilde{} Total\_Met\_Min + gender + Age, data =
pracdat.mod1, family = ``binomial'')

\begin{tabular}{l|r|r|r|r}
\hline
  & Estimate & Std. Error & z value & Pr(>|z|)\\
\hline
(Intercept) & 0.935 & 0.535 & 1.75 & 0.081\\
\hline
Total\_Met\_Min & 0.034 & 0.049 & 0.68 & 0.494\\
\hline
genderMale & -0.578 & 0.262 & -2.20 & 0.028\\
\hline
Age & -0.004 & 0.012 & -0.32 & 0.746\\
\hline
\end{tabular}

\includegraphics{test_latex_files/figure-latex/unnamed-chunk-7-1.pdf}

\textbf{If our specific aims are testing whether shift and total
metabolic minutes have an effect on weight gain, shouldn't the be in all
the models}

Multivariate Linear Model - only GAINED

\begin{verbatim}
##   c.cv.weight1..cv.weight3..cv.weight4.
## 1                                  0.39
## 2                                  0.39
## 3                                  0.38
\end{verbatim}

\begin{tabular}{l|r|r|r|r}
\hline
  & Estimate & Std. Error & t value & Pr(>|t|)\\
\hline
(Intercept) & 2.4299 & 0.2133 & 11.39 & 0.000\\
\hline
shift11am & -0.1790 & 0.1590 & -1.13 & 0.262\\
\hline
shift12pm & -0.1177 & 0.2370 & -0.50 & 0.620\\
\hline
shift1pm & -0.5157 & 0.2895 & -1.78 & 0.076\\
\hline
shift2pm & 0.4167 & 0.2270 & 1.84 & 0.068\\
\hline
shift7am & 0.3085 & 0.1850 & 1.67 & 0.097\\
\hline
shift8am & 0.3241 & 0.1262 & 2.57 & 0.011\\
\hline
shift9am & 0.0687 & 0.1481 & 0.46 & 0.643\\
\hline
shiftother & 0.1295 & 0.2096 & 0.62 & 0.537\\
\hline
Total\_Met\_Min & 0.0025 & 0.0184 & 0.13 & 0.894\\
\hline
Age & 0.0015 & 0.0041 & 0.38 & 0.707\\
\hline
genderMale & 0.1790 & 0.0979 & 1.83 & 0.069\\
\hline
\end{tabular}

\includegraphics{test_latex_files/figure-latex/unnamed-chunk-8-1.pdf}

Based on preliminary data analysis, we found shift is not a significant
predictor. We found that 7-8 am are significant, which is an interesting
perspective on morning versus afternoon shifts.

\hypertarget{discussion-and-conclusion}{%
\subsection{Discussion and Conclusion}\label{discussion-and-conclusion}}

\hypertarget{appendix}{%
\subsection{Appendix}\label{appendix}}

Test for normality of the continuous variables:

\includegraphics{test_latex_files/figure-latex/unnamed-chunk-9-1.pdf}

\begin{verbatim}
##      Age height pounds_gained bweight BMI VigTime WalkTime ModTime
## [1,] 269    189           105     133 284      87      115      91
## [2,] 179    288           122     331  83     282      206      80
##      Total_Met_Min Begin_bweight Before_BMI BMI_change
## [1,]            87           331        284        122
## [2,]           287           133         83        262
\end{verbatim}

\includegraphics{test_latex_files/figure-latex/unnamed-chunk-9-2.pdf}


\end{document}
